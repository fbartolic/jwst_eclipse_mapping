% Define document class
\documentclass[modern]{aastex631}

% Bibliography stuff
\bibliographystyle{aasjournal}
\setlength\LTcapwidth{\textwidth}

% Begin!
\begin{document}

% Title
\title{
    \vspace{-3em}
    \textbf{
        What can we learn about spatial maps of exoplanets in emitted light with 
        JWST photometric observations of secondary eclipses?
    }
}
% Author list
\author[0000-0001-8630-9794]{Fran Bartoli\'c}
\email{fb90@st-andrews.ac.uk}
\affil{Centre~for~Exoplanet~Science, SUPA, School~of~Physics~and~Astronomy, University~of~St.~Andrews, St.~Andrews, UK}
\affil{Center~for~Computational~Astrophysics, Flatiron~Institute, New~York, NY, USA}
\author[0000-0002-0296-3826]{Rodrigo Luger}
\author[0000-0002-9328-5652]{Daniel Foreman-Mackey}
\affil{Center~for~Computational~Astrophysics, Flatiron~Institute, New~York, NY, USA}
%

\begin{abstract}
   We test the feasibility of resolving spatially localized features on the daysides of tidally 
   locked exoplanets using secondary eclipse light curves in emitted light. 
We find that even with noiseless light curves and the assumption of fixed orbital parameters, there are 
fundamental degeneracies in the structure of inferred maps. 
With noisy light curves, differentiating between a homogenous spot and finer spatial 
features (due to, for instance, the presence of large storms or clouds in the atmosphere) is 
highly dependant on the size and the intensity contrast of the feature.
For a given surface map expanded in a basis of spherical harmonic, there is a signal-to-noise dependent 
threshold which needs to be reached so that the ingress and egress parts of the light 
curve start constraining higher order modes beyond $l=2$ or so.
If the data is too noisy, the information about the inferred map comes mostly from the phase curve 
signal and the eclipse depth.
Using temperature snapshots from a sophisticated hydrodynamics simulation of the Hot Jupiter HD189733b's
atmosphere displaying spatially variable planetary scale storms and vortices, 
we simulated eclipse light curves with the expected signal-to-noise ratio of JWST's NIRCam camera. 
We found that although there is very little hope of resolving maps 
with a resolution higher than $l=2$ ($\sim 90^\circ$), the expected variation in 
the eclipse depths and the phase curve signal due to the time variability of the weather at different 
epochs should still be detectable.
\end{abstract}

\section{Introduction}
\label{sec:introduction}
One of the most notable advances in exoplanetary science over the past decade has been the ability
to reconstruct coarse two-dimensional spatial maps of exoplanets using high precision phase 
curves and secondary eclipse observations.
\cite{knutson2007}, \cite{majeau2012b} and \cite{dewit2012} used Spitzer mid-infrared observations 
of secondary eclipses of the Hot Jupiter HD189733b and found that surface emission is best 
described by the presence of a large hot spot on the dayside of the planet which is 
longitudinally offset from the substellar point. 
Similarly, \cite{stevenson2014} produced temperature maps of the Hot Jupiter WASP-43b, 
\cite{demory2013} mapped the Hot Jupiter Kepler-7b in reflected light and \cite{demory2016a} 
mapped the thermal emission from the Super Earth 55 Cancri e, although all of these only 
captured longitudinal variations in intensity.

Real exoplanet atmospheres of are certain to have three-dimensional spatial 
inhomogeneities in emission more complex than a single hot spot due to the presence of clouds, zonal 
jets, storms, waves etc. \citep{showman2020}.
Recent high resolution simulations of Hot Jupiter atmospheres by \cite{cho2021} which were able 
to capture smaller scale turbulence show the presence of storms at a range of scales, including 
planetary scales, with quasi-periodic time variability and multiple equilibrium cycles.
These atmospheric phenomena should result in spatial variation in the intensity of emitted 
and reflected light across the visible disc of the planet which is potentially observable.

There have been significant advances in statistical modeling of phase curves and eclipse light
curves in recent years.
Most notably, \cite{luger2019a} introduced the \textsf{starry} algorithm which enables analytic
computation of phase curves and occultation light curves for bodies with arbitrary emission maps 
expressed in a spherical harmonic basis and\cite{luger2021d} expanded the algorithm for the (considerably more complicated) case of 
reflected light.

of bodies whose emission surface is 
expanded in a basis of spherical harmonics. 


In this paper, we investigate under which conditions it possible to recover more complex 
spatial features than large hot spots. 
We focus on emitted light from tidally locked planets and ignore uncertainties in the 
orbital elements.  
We start by describing the degeneracies inherent in the eclipse mapping problem even at 
infinite signal-to-noise and then investigate the relationship between noise and information 
content of phase curves and eclipse light curves. 


\section{Methods}
\label{sec:methods}

\subsection{The model}
\label{ssec:model}

We use the \textsf{starry} mapping framework (CITE) to compute eclipse light curves of a simulated planet with thermal surface emission.
\textsf{starry} enables fast computation of occultation (eclipse) light curves in both emitted and reflected light by analytically integrating the flux over the planet and the stellar disc 
Most importantly, \textsf{starry} 

\subsection{The nullspace}
\label{ssec:nullspace}

\begin{figure}[t!]
    \begin{centering}
    \includegraphics[width=\linewidth]{figures/preimages.pdf}
    \caption{
        The top row shows a collection of simulated spherical harmonic maps with different spatial 
        features and in the bottom row are the the corresponding \emph{preimages} -- maps 
        constructed only from those spherical harmonic
        coefficients which are not in the nullspace of the linear operator which 
        maps the 2D map into a 1D light curve. 
        The preimages represent the best-case scenario for reconstructing the original maps, they are the solution of the linear problem
        when the signal-to-noise for the light curve tends to infinity.
        To compute the preimages we consider a tidally locked Jupiter size planet in a 1 day orbit around a $1R_\odot$ star.
        Each row below the top row corresponds to a different value of the impact parameter of the secondary eclipse.
    }
    \label{fig:preimages}
    \end{centering}
\end{figure}

\section{Results}
\label{sec:results}


\subsection{A simple example -- localizing a hot spot}
\label{ssec:localizing_spot}


\begin{figure}[t!]
    \begin{centering}
    \includegraphics[width=\linewidth]{figures/varying_contrast_spot.pdf}
    \caption{
        Dependence of the inferred maps on the contrast $c$ between the feature and the background.
        The top row shows simulated maps at $l=25$ with a hot spot offset from the substellar point, located at
        $20^\circ$ latitude and $15^\circ$ longitude with a radius of $30^\circ$.
        Each column shows a map with a different spot contrast $c$ but all maps have the same dayside 
        flux ratio relative to the star, set to $0.001$.
        The second row shows the mean inferred maps together with a few sample maps from the posterior
        (small circles). 
        The grey cross marks the center of the simulated spot. 
        The bottom two rows show the simulated secondary eclipse light curves with the fitted flux 
        (solid orange line) and the residuals between the data and the simulated flux for maps 
        shown in the top row, except truncated to $l=1$ to emphasize that the scale of this difference 
        relative to the noise level determines the quality of the inferred maps.
        The solid lines show the binned residuals in 5 minute bins. 
    }
    \label{fig:varying_contrast_spot}
    \end{centering}
\end{figure}

\begin{figure}[t!]
    \begin{centering}
    \includegraphics[width=\linewidth]{figures/spot_residuals.pdf}
    \caption{
       Residuals between flux during egress computed with simulated maps at $l=25$ consisting of a single spot at 
       $0^\circ$ longitude with varying size, contrast and latitude (while holding the planet to star
       dayside flux ratio constant at $0.001$), and the flux computed with the same maps truncated to 
       $l=1$. 
       Since the spot is always at $0^\circ$ longitude, the ingress and egress flux is the same
       so we only show the egress.
       The deviation from the baseline model ($l=1$ maps) is maximized for large spots with large 
       contrasts at appreciable latitudes.
    }
    \label{fig:spot_residuals}
    \end{centering}
\end{figure}


\subsection{A more realistic example -- mapping weather patterns on Hot Jupiters}
\label{ssec:weather}

\begin{figure}[t!]
    \begin{centering}
    \includegraphics[width=\linewidth]{figures/hydro_sim_snapshots.pdf}
    \caption{
       Temperature snapshots from a hydrodynamics simulation of a Hot Jupiter atmosphere at different times, at a pressure of 
       1 bar. 
       The top row shows the the temperature maps at $l=25$ in  Mollweide projection and the bottom row shows the same maps truncated 
       to $l=2$. 
       The maps show the evolution of a pair of planetary scale storms -- a so called modon.
    }
    \label{fig:hydro_sim_snapshots:maps}
    \end{centering}
\end{figure}

\begin{figure}[t!]
    \begin{centering}
    \includegraphics[width=\linewidth]{figures/hydro_sim_snapshot_lightcurves.pdf}
    \caption{
        Predicted fluxes for simulated maps of HD189733b shown in \ref{fig:hydro_sim_snapshots:maps} assuming photometric observations 
        in the F444W $4.5\mu m$ JWST NIRCam filter.
        The top row shows the fluxes for each of the snapshots in addition to the mean flux across epochs. 
        The bottom row shows the difference in the predicted fluxes for the snapshots at $l=25$ (top row of 
        \ref{fig:hydro_sim_snapshots:maps}) and the predicted fluxes for the same snapshots truncated to $l=2$ (bottom row of 
        \ref{fig:hydro_sim_snapshots:maps}).
        The difference in flux for different simulation snapshots can be as 200 ppm (top row) but constraining maps 
        at a resolution higher than $l=2$ requires data with noise at the level of 10 ppm (bottom row). 
    }
    \label{fig:hydro_sim_snapshots:lightcurves}
    \end{centering}
\end{figure}

\begin{figure}[t!]
    \begin{centering}
    \includegraphics[width=\linewidth]{figures/hydro_sim_snapshots_fits_complete.pdf}
    \caption{
        Fits to the simulated maps shown in \ref{fig:hydro_sim_snapshots:maps} assuming photometric observations
        (including everything but the transit) in the F444W $4.5\mu m$ JWST NIRCam filter with exposure time of $3.7 s$ for two different values 
        of the signal-to-noise ratio on the secondary eclipse depth. 
        The top row shows the simulated maps at $l=25$, the middle row shows the mean inferred maps at $l=5$
        for S/N=20 which is roughly the expected performance of JWST for HD189733b assuming a single epoch
        observation, and the bottom row shows the inferred maps for S/N=100 which is the signal-to-noise ratio
        at which the deviations from an $l=2$ map shown in \ref{fig:hydro_sim_snapshots:lightcurves} start
        becoming significant.
        The mini maps beneath each mean inferred map are the posterior samples. 
    }
    \label{fig:hydro_sim_snapshots_fits:complete}
    \end{centering}
\end{figure}

\begin{figure}[t!]
    \begin{centering}
    \includegraphics[width=\linewidth]{figures/hydro_sim_snapshots_fits_eclipse_only.pdf}
    \caption{
        Same as Figure~\ref{fig:hydro_sim_snapshots_fits:complete} except we fit only the portions of complete
        light curves around the eclipse.
    }
    \label{fig:hydro_sim_snapshots_fits:eclipse_only}
    \end{centering}
\end{figure}



\section{Discussion and conclusions}
\label{sec:discussion}

\subsection{Summary}
\label{ssec:summary}

\subsection{Observability of spatial features on Hot Jupiters with JWST and LUVOIR}
\begin{figure}[t!]
    \begin{centering}
    \includegraphics[width=\linewidth]{figures/eclipse_snr.pdf}
    \caption{
       Estimates of the signal-to-noise ratio on the secondary eclipse for JWST 
        observations in the F444W $4.5\mu m$ filter for a Jupiter size planet 
        orbiting a Sun like star ($T_\mathrm{eff}=5000$K) as 
       a function of the planet equilibrium temperature and distance to the star (left panel).
       The right panel shows the same thing except we scale the collecting area to match the collecting 
       area of the planned LUVOIR-A telescope.
    }
    \label{fig:eclipse_snr}
    \end{centering}
\end{figure}


\subsection{Mapping spatial features in reflected light?}

\subsection{Spectral maps}

% Bibliography 
\bibliography{bib}


\appendix
\section{Appendix 1}
\clearpage

\end{document}