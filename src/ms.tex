% Define document class
\documentclass[modern]{aastex631}

% Bibliography stuff
\bibliographystyle{aasjournal}
\setlength\LTcapwidth{\textwidth}

% Begin!
\begin{document}

% Title
\title{
    \vspace{-3em}
    \textbf{
        What can we learn about exoplanet maps in emitted light from photometric observations of secondary eclipses using JWST?
    }
}
% Author list
\author[0000-0001-8630-9794]{Fran Bartoli\'c}
\email{fb90@st-andrews.ac.uk}
\affil{Centre~for~Exoplanet~Science, SUPA, School~of~Physics~and~Astronomy, University~of~St.~Andrews, St.~Andrews, UK}
\affil{Center~for~Computational~Astrophysics, Flatiron~Institute, New~York, NY, USA}
\author[0000-0002-0296-3826]{Rodrigo Luger}
\author[0000-0002-9328-5652]{Daniel Foreman-Mackey}
\affil{Center~for~Computational~Astrophysics, Flatiron~Institute, New~York, NY, USA}
%

\begin{abstract}
We investigate the degeneracies involved in the process of reconstructing spatial maps of 
exoplanets from secondary eclipse light curves in emitted light, and test the feasibility of 
spatially resolving localized features on the daysides of Hot Jupiters using JWST.
We find that even with noiseless light curves and assuming fixed orbital parameters, there are 
interesting degeneracies in the structure of inferred maps. 
With realistic light curves, differentiating between a homogenous spot and finer spatial 
structure (due to, for instance, the presence of large storms and clouds) is highly dependant 
on the intensity contrast between the feature and the background. 
There is a signal-to-noise dependent threshold which needs to be reached so that the ingress 
and egress parts of the light curve start constraining higher order modes of the surface map.
Without reaching this threshold there is no hope of constraining finer structure on the dayside
of the planet and the information content about the surface is dominated by the phase curve 
signal and the eclipse depth.
We find that even for the brightest Hot Jupiters, the signal-to-noise 
ratio of the simulated JWST light curves isn't high enough to provide much information about 
the surface beyond excluding the presence of very high contrast features.
\end{abstract}

\section{Introduction}
\label{sec:introduction}
One of the most notable advances in exoplanetary science over the past decade has been the ability
to reconstruct coarse two-dimensional spatial maps of exoplanets using high precision phase 
curves and secondary eclipse observations.
\cite{knutson2007}, \cite{majeau2012b} and \cite{dewit2012} used Spitzer mid-infrared observations 
of secondary eclipses of the Hot Jupiter HD189733b and found that surface emission is best 
described by the presence of a large hot spot on the dayside of the planet which is 
longitudinally offset from the substellar point. 
Similarly, \cite{stevenson2014} produced temperature maps of the Hot Jupiter WASP-43b, 
\cite{demory2013} mapped the Hot Jupiter Kepler-7b in reflected light and \cite{demory2016a} 
mapped the thermal emission from the Super Earth 55 Cancri e, although all of these only 
captured longitudinal variations in intensity.

Real exoplanet atmospheres of are certain to have three-dimensional spatial 
inhomogeneities in emission more complex than a single hot spot due to the presence of clouds, zonal 
jets, storms, waves etc. \citep{showman2020}.
Recent high resolution simulations of Hot Jupiter atmospheres by \cite{cho2021} which were able 
to capture smaller scale turbulence show the presence of storms at a range of scales, including 
planetary scales, with quasi-periodic time variability and multiple equilibrium cycles.
These atmospheric phenomena should result in spatial variation in the intensity of emitted 
and reflected light across the visible disc of the planet which is potentially observable.

There have been significant advances in statistical modeling of phase curves and eclipse light
curves in recent years.
Most notably, \cite{luger2019a} introduced the \textsf{starry} algorithm which enables analytic
computation of phase curves and occultation light curves for bodies with arbitrary emission maps 
expressed in a spherical harmonic basis and\cite{luger2021d} expanded the algorithm for the (considerably more complicated) case of 
reflected light.

of bodies whose emission surface is 
expanded in a basis of spherical harmonics. 


In this paper, we investigate under which conditions it possible to recover more complex 
spatial features than large hot spots. 
We focus on emitted light from tidally locked planets and ignore uncertainties in the 
orbital elements.  
We start by describing the degeneracies inherent in the eclipse mapping problem even at 
infinite signal-to-noise and then investigate the relationship between noise and information 
content of phase curves and eclipse light curves. 


\section{Methods}
\label{sec:methods}

\subsection{Model}
\label{ssec:model}
We use the \textsf{starry} mapping framework (CITE) to compute eclipse light curves of a simulated planet with thermal surface emission.
\textsf{starry} enables fast computation of occultation (eclipse) light curves in both emitted and reflected light by analytically integrating the flux over the planet and the stellar disc 
Most importantly, \textsf{starry} 


To compute eclipse light curves 
\section{Results}
\label{sec:results}

\subsection{Degeneracies}
\label{ssec:degeneracies}


\begin{figure}[t!]
    \begin{centering}
    \includegraphics[width=\linewidth]{figures/preimages.pdf}
    \caption{}
    \label{fig:preimages}
    \end{centering}
\end{figure}





\section{Discussion}
\label{sec:discussion}

\begin{figure}[t!]
    \begin{centering}
    \includegraphics[width=\linewidth]{figures/varying_contrast_spot.pdf}
    \caption{}
    \label{fig:varying_contrast_spot}
    \end{centering}
\end{figure}

\begin{figure}[t!]
    \begin{centering}
    \includegraphics[width=\linewidth]{figures/spot_residuals.pdf}
    \caption{}
    \label{fig:spot_residuals}
    \end{centering}
\end{figure}

\begin{figure}[t!]
    \begin{centering}
    \includegraphics[width=\linewidth]{figures/hydro_sim_snapshots.pdf}
    \caption{}
    \label{fig:hydro_sim_snapshots:maps}
    \end{centering}
\end{figure}

\begin{figure}[t!]
    \begin{centering}
    \includegraphics[width=\linewidth]{figures/hydro_sim_snapshot_lightcurves.pdf}
    \caption{}
    \label{fig:hydro_sim_snapshots:lightcurves}
    \end{centering}
\end{figure}

\begin{figure}[t!]
    \begin{centering}
    \includegraphics[width=\linewidth]{figures/hydro_sim_snapshots_fits.pdf}
    \caption{}
    \label{fig:hydro_sim_snapshots_fits}
    \end{centering}
\end{figure}

\begin{figure}[t!]
    \begin{centering}
    \includegraphics[width=\linewidth]{figures/eclipse_snr.pdf}
    \caption{}
    \label{fig:eclipse_snr}
    \end{centering}
\end{figure}



% Bibliography 
\bibliography{bib}


\appendix
\section{Appendix 1}
\clearpage

\end{document}